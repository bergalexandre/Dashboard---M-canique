\textbf{\large Ch\^assis}\\\
\begin{tabularx}{\linewidth}{
    |>{\hsize=2.5\hsize}X|% 10% of 4\hsize 
    >{\hsize=0.5\hsize}X|% 30% of 4\hsize
    >{\hsize=0.5\hsize}X|% 30% of 4\hsize
    >{\hsize=0.5\hsize}X|% 30% of 4\hsize
       % sum=0.2\hsize for 4 columns
  }
    \hline
    \textbf{Objectif} & \textbf{Responsable} & \textbf{\% attendu} & \textbf{\% fait}
    \\\hline
        Intégration des sous-systèmes (suspension, direction, transmission, moteur). & A.M., D.B., M.R. & 10\% & 12.5\% \\\hline 
       Déterminer les matériaux et services nécessaires pour commandite Gurit et approximer la masse. & A.M., D.B., M.R.& 60\% & 60\% \\\hline 
       Études FEA pour les cas de chargements. & Donald & 25\% & 15\% \\\hline
       
\end{tabularx}



\hfill \break
\textbf{\large Coque}\\
\begin{tabularx}{\linewidth}{
    |>{\hsize=2.5\hsize}X|% 10% of 4\hsize 
    >{\hsize=0.5\hsize}X|% 30% of 4\hsize
    >{\hsize=0.5\hsize}X|% 30% of 4\hsize
    >{\hsize=0.5\hsize}X|% 30% of 4\hsize
  }
    \hline
    \textbf{Objectif} & \textbf{Responsable}  & \textbf{\% attendu} & \textbf{\% fait} \\\hline
       Terminer le modèle CFD et le représenter à l'expert & Joé, Charles, J-S & 100\% & 90\%\\\hline
       Terminer simulations sur devant du véhicule + plaque vs garde boue sur roues& Joé, Chalres, J-S & 100\% & 100\%\\\hline
       Terminer simulations sur derrière du véhicule + validations sur la fabricabilité & Joé, Chalres, J-S & 0\% & 10\%\\\hline
\end{tabularx}



\hfill \break
\textbf{\large Simulateur}\\
\begin{tabularx}{\linewidth}{
     |>{\hsize=2.5\hsize}X|% 10% of 4\hsize 
    >{\hsize=0.5\hsize}X|% 30% of 4\hsize
    >{\hsize=0.5\hsize}X|% 30% of 4\hsize
    >{\hsize=0.5\hsize}X|% 30% of 4\hsize
  }
    \hline
    \textbf{Objectif} & \textbf{Responsable}  & \textbf{\% attendu} & \textbf{\% fait} \\\hline
       Courbes de l'effet de la résistance au roulement et du coefficient aéro.& Alex & 50\%& 100\% \\\hline 
\end{tabularx}



\hfill \break
\textbf{\large Suspension}\\
\begin{tabularx}{\linewidth}{
    |>{\hsize=2.5\hsize}X|% 10% of 4\hsize 
    >{\hsize=0.5\hsize}X|% 30% of 4\hsize
    >{\hsize=0.5\hsize}X|% 30% of 4\hsize
    >{\hsize=0.5\hsize}X|% 30% of 4\hsize
  }
    \hline
    \textbf{Objectif} & \textbf{Responsable}  & \textbf{\% attendu} & \textbf{\% fait} \\\hline
       Terminer les cas de chargement en utilisation normale et en fatigue.&Alex  & 0\%& 50\%\\\hline 
       DCL et calculs analytiques des pièces de suspension (Matlab).&Alex & 50\%& 50\% \\\hline
       Choix des roulements avec SKF &Alex  & 0\%& 0\% \\\hline  
       Début des FEA de la suspension &Alex, Jérémi & 50\% & 50\% \\\hline  
       Conception initiale des roues. & Jérémi & 66\% & 50\% \\\hline
       & \\\hline 
\end{tabularx}



\hfill \break
\textbf{\large Freins}\\
\begin{tabularx}{\linewidth}{
    |>{\hsize=2.5\hsize}X|% 10% of 4\hsize 
    >{\hsize=0.5\hsize}X|% 30% of 4\hsize
    >{\hsize=0.5\hsize}X|% 30% of 4\hsize
    >{\hsize=0.5\hsize}X|% 30% of 4\hsize
  }
    \hline
    \textbf{Objectif} & \textbf{Responsable}  & \textbf{\% attendu} & \textbf{\% fait} \\\hline
       & \\\hline 
\end{tabularx}


\hfill \break
\textbf{\large Ergonomie}\\
\begin{tabularx}{\linewidth}{
    |>{\hsize=2.5\hsize}X|% 10% of 4\hsize 
    >{\hsize=0.5\hsize}X|% 30% of 4\hsize
    >{\hsize=0.5\hsize}X|% 30% of 4\hsize
    >{\hsize=0.5\hsize}X|% 30% of 4\hsize
  }
    \hline
    \textbf{Objectif} & \textbf{Responsable}  & \textbf{\% attendu} & \textbf{\% fait} \\\hline
       Commencer la conception de certains composants et supports (phase 1 de 3) & Gab, Joé &30 \% &10\% \\\hline
       Conception du volant - intégration boutons et écran - Choix de matériaux & Joé &50 \% &30\% \\\hline       

\end{tabularx}

\hfill \break
\textbf{\large Direction}\\
\begin{tabularx}{\linewidth}{
    |>{\hsize=2.5\hsize}X|% 10% of 4\hsize 
    >{\hsize=0.5\hsize}X|% 30% of 4\hsize
    >{\hsize=0.5\hsize}X|% 30% of 4\hsize
    >{\hsize=0.5\hsize}X|% 30% of 4\hsize
  }
    \hline
    \textbf{Objectif} & \textbf{Responsable}  & \textbf{\% attendu} & \textbf{\% fait} \\\hline
        Faire fiter le système direction avec les autres systèmes et s'assurer d'avoir un bon ackermann.&Gabriel  & 60\% & 70\%
        \\\hline 
        DCL et calculs analytiques des pièces de la direction &Gabriel  & 20\% & 40\%
        \\\hline
        Début des FEA direction  &Gabriel  & 10\% & 10\%
        \\\hline
\end{tabularx}


% --------------------------------------------------------------------------------
% Copy/Paste following code to add a new sub-system
% --------------------------------------------------------------------------------
% \hfill \break
% \textbf{\large SOUS_SYSTÈME}\\
% \begin{tabularx}{\linewidth}{
%     |>{\hsize=1.75\hsize}X|% 10% of 4\hsize 
%     >{\hsize=0.25\hsize}X|% 30% of 4\hsize
%       % sum=0.2\hsize for 4 columns
%   }
%     \hline
%     \textbf{Objectif} & \textbf{Responsable} \\\hline
%       & \\\hline 
%       & \\\hline
%       & \\\hline 
% \end{tabularx}
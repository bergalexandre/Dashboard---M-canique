\textbf{\large Ch\^assis}\\
\begin{tabularx}{\linewidth}{
    |>{\hsize=1.75\hsize}X|% 10% of 4\hsize 
    >{\hsize=0.25\hsize}X|% 30% of 4\hsize
       % sum=0.2\hsize for 4 columns
  }
    \hline
    \textbf{Objectif} & \textbf{Responsable} \\\hline
       Conception détaillé du ch\^assis (dossier, poutres et plancher). & A.M., D.B., M.R.\\\hline 
       Études FEA pour les cas de chargements. & Donald\\\hline
       Intégration des sous-systèmes (suspension, direction, ). & A.M., M.R. \\\hline 
\end{tabularx}



\hfill \break
\textbf{\large Coque}\\
\begin{tabularx}{\linewidth}{
    |>{\hsize=1.75\hsize}X|% 10% of 4\hsize 
    >{\hsize=0.25\hsize}X|% 30% of 4\hsize
       % sum=0.2\hsize for 4 columns
  }
    \hline
    \textbf{Objectif} & \textbf{Responsable} \\\hline
       Terminer le modèle CFD et le représenter à l'expert & Joé, Charles, J-S \\\hline
       & \\\hline
       & \\\hline 
\end{tabularx}



\hfill \break
\textbf{\large Simulateur}\\
\begin{tabularx}{\linewidth}{
    |>{\hsize=1.75\hsize}X|% 10% of 4\hsize 
    >{\hsize=0.25\hsize}X|% 30% of 4\hsize
       % sum=0.2\hsize for 4 columns
  }
    \hline
    \textbf{Objectif} & \textbf{Responsable} \\\hline
       Modélisation des pneus, bearings, freins et de la résistance en virage.&Alex \\\hline 
       Nettoyage du script Matlab et Simulink.&Alex \\\hline
       Mettre à jour le guide d'équations mécaniques.&Alex \\\hline 
\end{tabularx}



\hfill \break
\textbf{\large Suspension}\\
\begin{tabularx}{\linewidth}{
    |>{\hsize=1.75\hsize}X|% 10% of 4\hsize 
    >{\hsize=0.25\hsize}X|% 30% of 4\hsize
       % sum=0.2\hsize for 4 columns
  }
    \hline
    \textbf{Objectif} & \textbf{Responsable} \\\hline
       Réunion avec les systèmes châssis et freins pour discuter de l'intégration avant de revoir le design conceptuel.&Alex \\\hline 
       Recherche, puis choix ou conception pour les roues. & Jérémi\\\hline
       & \\\hline 
\end{tabularx}



\hfill \break
\textbf{\large Freins}\\
\begin{tabularx}{\linewidth}{
    |>{\hsize=1.75\hsize}X|% 10% of 4\hsize 
    >{\hsize=0.25\hsize}X|% 30% of 4\hsize
       % sum=0.2\hsize for 4 columns
  }
    \hline
    \textbf{Objectif} & \textbf{Responsable} \\\hline
     Conception du frein de stationnement. & Jérémi\\\hline 
       & \\\hline
       & \\\hline 
\end{tabularx}


\hfill \break
\textbf{\large Ergonomie}\\
\begin{tabularx}{\linewidth}{
    |>{\hsize=1.75\hsize}X|% 10% of 4\hsize 
    >{\hsize=0.25\hsize}X|% 30% of 4\hsize
       % sum=0.2\hsize for 4 columns
  }
    \hline
    \textbf{Objectif} & \textbf{Responsable} \\\hline
       Terminer la conception et démarrer l'impression du support des batteries & Charles\\\hline 
       & \\\hline
       & \\\hline 
\end{tabularx}

\hfill \break
\textbf{\large Direction}\\
\begin{tabularx}{\linewidth}{
    |>{\hsize=1.75\hsize}X|% 10% of 4\hsize 
    >{\hsize=0.25\hsize}X|% 30% of 4\hsize
       % sum=0.2\hsize for 4 columns
  }
    \hline
    \textbf{Objectif} & \textbf{Responsable} \\\hline
        Faire fiter le système direction avec les autres systèmes et s'assurer d'avoir un bon ackermann.&Gabriel
        \\\hline 
        Faire un assemblage pour voir les interférences et l'angle des roues.&Gabriel
        \\\hline
       & \\\hline 
\end{tabularx}


% --------------------------------------------------------------------------------
% Copy/Paste following code to add a new sub-system
% --------------------------------------------------------------------------------
% \hfill \break
% \textbf{\large SOUS_SYSTÈME}\\
% \begin{tabularx}{\linewidth}{
%     |>{\hsize=1.75\hsize}X|% 10% of 4\hsize 
%     >{\hsize=0.25\hsize}X|% 30% of 4\hsize
%       % sum=0.2\hsize for 4 columns
%   }
%     \hline
%     \textbf{Objectif} & \textbf{Responsable} \\\hline
%       & \\\hline 
%       & \\\hline
%       & \\\hline 
% \end{tabularx}
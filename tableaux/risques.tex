\begin{tabularx}{\linewidth}{
    |>{\hsize=0.40\hsize}X|% 10% of 4\hsize 
    >{\hsize=0.25\hsize}X|% 30% of 4\hsize
    >{\hsize=0.25\hsize}X|% 30% of 4\hsize 
    >{\hsize=0.1\hsize}X|% 30% of 4\hsize 
       % sum=0.2\hsize for 4 columns
  }
    \hline
    \textbf{Risque} & \textbf{Mitigation} & \textbf{Conséquence} & \textbf{Priorité}\\\hline
     Simulation sans modèle de moteur.&Faire le modèle moteur Simulink (Élec, en cours).&Avoir un modèle Matlab complet. & 3/5\\\hline
     Propriétés exactes des matériaux composites inconnus.&Sélection de matériaux avec des propriétés mieux connus et utiliser des facteurs de sécurité appropriés.&Recherche pour trouver les propriétés et essais en laboratoire si nécessaire. & 3/5\\\hline
     &  &  & \\\hline     
  \end{tabularx}
  
  
% Template pour le tableau des risques de la semaine : 

% Risque : le risque de la semaine
% Mitigation: comment mitiger le risque pour réduire ses conséquences/occurances/etc.
% Conséquence : La ou les conséquences si le risque survient
% Priorité : Priorité du risque sur une échelle de 1 à 5.  5 étant + prioritaire. La priorité est basé sur les conséquences la probablité d'occurence, etc.  
  
%  \textbf{Risque} & \textbf{Mitigation} & \textbf{Conséquence} & \textbf{Priorité}\\\hline
%    Risque & Mitigation & Conséquence & Priorité\\\hline
%    Risque & Mitigation & Conséquence & Priorité\\\hline
%    Risque & Mitigation & Conséquence & Priorité\\\hline
%    Risque & Mitigation & Conséquence & Priorité\\\hline
%    Risque & Mitigation & Conséquence & Priorité\\\hline
%    Risque & Mitigation & Conséquence & Priorité\\\hline
%    Risque & Mitigation & Conséquence & Priorité\\\hline
%    Risque & Mitigation & Conséquence & Priorité\\\hline
%    Risque & Mitigation & Conséquence & Priorité\\\hline
%    Risque & Mitigation & Conséquence & Priorité\\\hline
%    Risque & Mitigation & Conséquence & Priorité\\\hline
%    Risque & Mitigation & Conséquence & Priorité\\\hline